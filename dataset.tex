Our work is based on a dataset of the British newspaper \textit{The Guardian} that contains approximately $626$K article URLs with $61.5$M corresponding comments from $1.25$M authors. 
The comment data provided us with the comment itself, the timestamp when the comment was posted, a reference of the parent comment, and the number of upvotes. Additionally, we had the author's username and a dedicated reference.

We assume that the number of comments of a given article is an essential feature to predict the comment volume of similar articles in the future. Fortunately, this feature could be simply determined by counting the corresponding comments for each article. Afterwards, we were able to decide if a given article was in the top $10$\% of the most commented articles within its released week.

\subsection{Data Cleaning}
Analysing the available data, we noticed that the distribution of the number of comments exhibited a bizarre peak at exactly $50$ comments per article. Therefore, we decided not taking these articles into account which reduced our dataset about $2$\% of the given articles.

To avoid anomalies using features that we derived from the articles, e.g. the publication time, we removed all articles that were released on 29th of February.

\subsection{Data enrichment}
To enrich our dataset, we use the offical \textit{Guardian API} to get the article text and further attributes like the category, the headline, and the publication time.
Due to API restrictions we were not able to download all articles which is way our dataset was reduced to $563$K articles.
Furthermore, we extracted metadata like the headline and article word count, the time data describing the day of the week, the day of the year, the hour and the minute of the publication date.

To take into account that articles with similar topic and publication time will share the amount of received comments for each article we add an additional feature named \textit{competitive score}. 

COMPETIVE SCORE HERE

For the following we call our final cleaned and enriched dataset the Guardian Article and Comment Corpus (GACC). It contains approximate $547$ K articles with $58.6$ million comments. 

Between 2006 and 2017 $1.1$ million users wrote at least one comment, $22$\% more than $10$, and $6$\% over $100$.

The GACC uses one out of $79$ categories for each article. In total, $18$\% of all articles are published under the category \textit{comment is free} which is used to debate contentious issues. Consequently, this category is also the most commented one. 
Each of the others, e.g. \textit{sport}, \textit{music} and \textit{politics} covers less than $7$\% of the GACC.