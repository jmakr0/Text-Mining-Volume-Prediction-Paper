\documentclass[10pt,sigconf]{acmart}

\usepackage[utf8]{inputenc}
\usepackage{booktabs} % For formal tables
\usepackage{array}
\usepackage{commath}
\newcommand*\rotbf[1]{\rotatebox{90}{\textbf{#1}}}
\newcommand{\specialcell}[2][c]{\begin{tabular}[#1]{@{}l@{}}#2\end{tabular}}
\newcommand{\specialcellbold}[2][c]{%
  \bfseries
  \begin{tabular}[#1]{@{}l@{}}#2\end{tabular}%
}

% Copyright
%\setcopyright{none}
%\setcopyright{acmcopyright}
%\setcopyright{acmlicensed}
% \setcopyright{rightsretained}
%\setcopyright{usgov}
%\setcopyright{usgovmixed}
%\setcopyright{cagov}
%\setcopyright{cagovmixed}

\begin{document}
\title{Text Mining in Practice: Comment Volume Prediction}
\subtitle{Extended Abstract} % do we need that?

\author{Johannes M. Kroschewski}
\affiliation{%
	\institution{Hasso-Plattner-Institut,\\ University of Potsdam}
	\streetaddress{Prof.-Dr.-Helmert-Str. 2–3}
	\city{Potsdam} 
	\state{Germany} 
	\postcode{14482}
}
\email{johannes.kroschewski@student.hpi.de}

\author{Friedrich C. Schöne}
\affiliation{%
	\institution{Hasso-Plattner-Institut,\\ University of Potsdam}
	\streetaddress{Prof.-Dr.-Helmert-Str. 2–3}
	\city{Potsdam} 
	\state{Germany} 
	\postcode{14482}
}
\email{friedrich.schoene@student.hpi.de}

\author{Nils H. Straßenburg}
\affiliation{%
  	\institution{Hasso-Plattner-Institut,\\ University of Potsdam}
	\streetaddress{Prof.-Dr.-Helmert-Str. 2–3}
	\city{Potsdam} 
	\state{Germany} 
	\postcode{14482}
}
\email{nils.strassenburg@student.hpi.de}

% The default list of authors is too long for headers}
\renewcommand{\shortauthors}{M. Kroschewski et al.}


\begin{abstract}
	abstract text \footnote{abstract footnote}
\end{abstract}
%
% The code below should be generated by the tool at
% http://dl.acm.org/ccs.cfm
% Please copy and paste the code instead of the example below. 
%
\begin{CCSXML}
	% TODO
<ccs2012>
 <concept>
  <concept_id>10010520.10010553.10010562</concept_id>
  <concept_desc>Computer systems organization~Embedded systems</concept_desc>
  <concept_significance>500</concept_significance>
 </concept>
 <concept>
  <concept_id>10010520.10010575.10010755</concept_id>
  <concept_desc>Computer systems organization~Redundancy</concept_desc>
  <concept_significance>300</concept_significance>
 </concept>
 <concept>
  <concept_id>10010520.10010553.10010554</concept_id>
  <concept_desc>Computer systems organization~Robotics</concept_desc>
  <concept_significance>100</concept_significance>
 </concept>
 <concept>
  <concept_id>10003033.10003083.10003095</concept_id>
  <concept_desc>Networks~Network reliability</concept_desc>
  <concept_significance>100</concept_significance>
 </concept>
</ccs2012>  
\end{CCSXML}

\ccsdesc[500]{Computer systems organization~Embedded systems}
\ccsdesc[300]{Computer systems organization~Redundancy}
\ccsdesc{Computer systems organization~Robotics}
\ccsdesc[100]{Networks~Network reliability}

% We no longer use \terms command
%\terms{Theory}

\keywords{text mining}

\maketitle

\section{Introduction}
% \subsection{Motivation and Goal}

% \subsection{Related Work}


\section{Dataset}
The dataset is taken from the British newspaper \textit{The Guardian} that contains approximately $626$K article URLs with $61.5$M corresponding comments from $1.25$M authors.
The comment data provided us with the comment text itself, the article reference, the author, the creation timestamp, a reference of the parent comment, and the number of upvotes. Additionally, we had the comment author's username and a dedicated reference.

The comments were posted between 2006 and 2017. All $1.25$M users wrote at least one comment, $22$\% of them more than $10$, and $6$\% over $100$.
The given articles are published in one out of $79$ categories. 
Articles featuring contentions issues are published in the \textit{comment is free} category, accounting $18$\% of all articles released.
Consequently, this category is also the most commented one. 
Each of the others, e.g.\ \textit{sport}, \textit{music}, and \textit{politics} cover less than $7$\% of the given dataset.

\subsection{Data Cleaning}
Analysis of the dataset revealed that the distribution of the number of comments per article exhibited an anomalous peak at exactly $50$ comments. 
Therefore, these articles were excluded which reduced our dataset size by about $2$\%.

To avoid anomalies using features that we derived from the article's publication time, we removed a small number of articles that were released on the 29th of February.

\subsection{Data Enrichment}
To enrich our dataset, we used the official \textit{Guardian API} \footnote{\url{https://open-platform.theguardian.com}} which provided the article text and further attributes such as the category, the headline, and the publication date.
Due to API restrictions, it was not possible to download all the articles hence why the number of articles was reduced by about $11$\%.
Furthermore, we extracted metadata such as the headline and article word count, and derived time features such as the day of the year, the day of the week, hour, and minute from the publication date.

Articles released at a similar time and discussing a similar topic will share their comment volume. To take this into account, a \textit{competitive score} was developed as an additional feature:

\begin{equation} \label{eq:competitive_score}
	compet_i = \sum_{n=1}^{j} \frac{\sigma'(t_i - t_j)}{\norm{\overrightarrow{a_i} - \overrightarrow{{a_j}}}^2}, i \neq j
\end{equation}

\begin{flalign*}
	\sigma&: \text{sigmoid function} & \\
	t_i&: \text{publication time of article } i & \\
	\overrightarrow{a_i}&: \text{\textit{Doc2Vec} vector of article } i \text{ text}& \\
\end{flalign*}

The numerical value of $compet_i$ describes how much a given article $i$ competes with all other articles.
Thereby, the difference between the publication time of the given articles shows how close they are published to each other.
In practice, articles are often published at the same time which would result in a time difference of zero.
Application of the derivate of the sigmoid function in the numerator accounts for this.
In order to identify articles concerning similar topics, the \textit{Doc2Vec} algorithm \cite{le2014doc2vec} was trained on our article corpus and the Euclidean distance is used to calculate how similar articles are to each other.
The distance in the denominator is squared to increase the impact of the article similarity.

To decide if a given article was in the top $10$\% of the most commented on articles within its released week was determined by using the number of comments for that article.

Throughout this paper, this cleaned and enriched dataset is referred to as the Guardian Article and Comment Corpus (GACC) which contains approximately $547$K articles with $58.6$M comments.



\section{Methodology}
To evaluate the quality of our features, we've implemented different architectures to compare them with each other.
All seven models use single or related features as input and generate a binary classification output using sigmoid as the activation function. Each of the models has an identifier which will be used as reference within the further text.

\subsection{Architectures}

\paragraph{Model 1} 
This model takes the article headline as input. We normalized the headline length and embedded the words using an embedding layer which we initialized with pretrained glove embeddings \cite{pennington2014glove}. The model uses a dense and batch normalization layer as hidden layer.

\paragraph{Model 2} 
Similar to \textit{Model 1}, this model takes the article headline as input. 
A convolutional layer with kernel sizes one, tree, and five is used instead of a dense layer as well as a pooling layer as proposed by Y. Kim \cite{kim2014convolutional}.

\paragraph{Model 3} 
This model takes the first $50$ words of the article text. They are embedded the same way as in \textit{Model 1} and \textit{Model 2} but are processed through an LSTM layer outputting the last cell state.

\paragraph{Model 4} 
This model takes the article category as input. The category reference gets embedded using an embedding layer and processed through a dense and a batch normalization layer.

\paragraph{Model 5} 
This model takes temporal features of the article's publication date as input. The features are the minute, hour, the day of the week, and day of the year. The features get processed the same way as in \textit{Model 4}.

\paragraph{Model 6} 
This model takes the word count of the headline and the article. The logarithm is calculated for both of them and used to create exponential sized bins for different lengths. Each logarithm gets embedded and processed like in \textit{Model 5} and \textit{Model 6}.

\paragraph{Model 7} 
This model takes the competitive score \ref{eq:competitive_score} and processes it through a dense layer as well as a batch normalization layer.

\subsection{Training}
We split the dataset into training, validation, and test sets using a $0.70/0.15/0.15$ distribution. Thus, we're using time-wise split by sorting with respect to the publication date.

Due to a strongly imbalanced training set, we use class weights to penalise our models in case of a wrong classification. They influence the weighting of the loss while training indirectly proportional to the class size of a giving training sample.

\subsection{Combined Architectures}
We combine several models to use multiple features and to improve our results. Thereby, the combined models share their classification layers but not the hidden layers.
The decision on whether we combine specific models is determined on their performance and on how much they correlate with each other. Combining these two metrics, we had a large set of models we were able to choose from. The correlations between each model pair are shown in Figure \ref{fig:correlation_matrix}. 

\begin{figure}
	\includegraphics[width=0.35\textwidth]{fig/correlations.png}
	\caption{\textmd{Correlation matrix of the basic models, using the test data set.}}
	\label{fig:correlation_matrix}
\end{figure}


\section{Evaluation}
We use \textit{precision} and \textit{recall} to evaluate our models. 
In practice, each article which produces a high amount of comments is more likely to bind resources of comment moderators and editors. Therefore, a high classification precision is important as well as a high recall to omit as few as possible.
Precision and recall are equally relevant and consequently, we use the harmonic mean, the $F_1$-score, of both metrics. As \textit{precision}, \textit{recall}, and the $F_1$-score are common metrics for binary classification problems, we are enabled to compare our results against existing approaches.

\subsection{Model Performances}
The overview of our results in \autoref{tbl:results_basic} depicts that all our models reach a significantly higher \textit{recall} than \textit{precision}.

\textit{Model 3} reaches the highest \textit{precision} with a value of $0.228$, and \textit{Model 7} the highest \textit{recall} with $0.917$. \textit{Model 4} is the best overall performing model with an $F_1$-score of $0.320$. 
%\textit{Model 1} to \textit{3} also performs relatively well with an $F_1$-scores of over $0.3$.

\textit{Model 1} and \textit{Model 2} process the same input. Despite the fact that the second model has approximately a third of the weights in its hidden layers, both models exhibit a similar performance for all three metrics.
We assume that the convolutional network can learn fewer features but has a much higher generalization potential through its shared weights.

\textit{Model 3} takes the first $50$ words of the article text as input. We discovered that using more or even all words has no positive effect on the model's predictions.

Compared to \textit{Model 1} and \textit{2} it reaches a worse \textit{recall} but a slightly higher \textit{precision} which leads to a similar $F_1$-score.
We suppose this occurs because of a similar context complexity which is provided by the headline and the first words of the article.

\textit{Model 4} achieves the best results. This seems surprising due to the low complexity of this feature but the relatively good results can be justified with the distribution of the categories within the GACC. 

FIGURE?

Articles assigned to the most used category have a probability of $22.8\%$ to be a top article. Classifying just these articles as a top article already leads to an $F_1$-score of $0.292$. If we only use the four categories with the highest percentage of top articles, leads to an $F_1$-score of $0.320$ which is the same result as achieved by this model.


\textit{Model 5} is the worst performing model. Its precision of $0.100$ is just as good as a random prediction.

FIGURE?

One reason is that $19$\% of all articles got released on full hours and might also be due to the fact that \textit{The Guardian} has an international readership spread over multiple time zones. Additionally, the day of the week has no significant effect on the weekly article performance. Therefore, this feature has almost no effect on the number of comments that an article receives. 

\textit{Model 6} had also a small but slightly better performance than \textit{Model 5}. 
Either the length of the headline or the length of the article text is feasible for an accurate prediction.

\textit{Model 7} processes just a single numeric value as input. 

For a good performance using the competitive score it would be necessary to be able to divide the values into distinct intervals that can be assigned to each class.
Unfortunately, the competitive score distribution of top articles is similar to the overall distribution of the competitive score.

% Table 3: Precision (P), recall (R), and F1-score of the baseline, all article and metadata features, annotations of comments shown on the first page, and all combined.
\begin{table}[h]
	\centering
	\caption{\textmd{Precision (P), recall (R), and $F_1$-score of all basic models.}}
	\label{tbl:results_basic}
	\vspace{-0.2cm}
	\begin{tabular}{cllccc}
		\toprule
		% \textbf{Features} &
		\specialcellbold{ID} &
		\specialcellbold{Input} &
		\specialcellbold{Type} &
		\specialcellbold{P} &
		\specialcellbold{R} &
		\specialcellbold{F$_1$} \\
		\midrule
		1 & headline & FC & .216 & .606  & .309  \\
		2 & headline & CNN & .221 & .603  & .314  \\
		3 & article & LSTM  & .228 & .493  & .302  \\
		4 & category & FC & .227 & .603  & .320  \\
		5 & time & FC  & .100 & .457  & .155  \\
		6 & text metrics & FC  & .133 & .738  & .222  \\
		7 & competitive score & FC & .112 & .917  & .197  \\
		\bottomrule
	\end{tabular}
\end{table}

\subsection{Combined Models}
As seen in the correlation matrix (Figure \ref{fig:correlation_matrix}) \textit{Model 1} and \textit{Model 2} correlate the most with a correlation value of 0.589. The same feature input explains this behaviour. The correlation between \textit{Model 3} and both \textit{Model 1} and \textit{Model 2} is also relatively high with the values $0.289$ and $0.255$ respectively. A reason for this could be that a headline and the first 50 words of an article transport a very similar context.

\textit{Model 2} is the best model using text therefore we decided to combine it with each other model. For most other models the correlation with \textit{Model 2} is low which also justifies this decision.
\textit{Model 5} which is correlating least with all other models has a very low $F_1$ score. Therefore we excluded it from our consideration to combine it with every other model.

The models, that we created combining \textit{Model 2} with each \textit{Model 5}, \textit{6} and \textit{7} performed almost the same as \textit{Model 2} itself. Therefore, we figured that publishing time, text metrics and competitive score are unqualified features for the headline text.

Combining each \textit{Model 3} and \textit{Model 4} with \textit{Model 2} showed small improvements in performance. In consequence we combined \textit{model 2}, \textit{Model 3} and \textit{Model 4} and got our best results. The better $F_1$-score of $0.357$ was due to the improvements of the precision value and a consistent recall.

Our $F_1$-score was $37\%$ higher than the presented value from Tsagikas et al. \cite{tsagkias2009predicting} and $24\%$ smaller than the presented value from Ambroselli et al. \cite{ambroselli2018prediction}.

BEGRÜNDUNG \& RECHTFERTIGUNG

\begin{table}[h]
\centering
\label{tbl:results_combined}
\caption{\textmd{Precision (P), recall (R), and F$_1$-score of all combined models.}}
\vspace{-0.2cm}\begin{tabular}{cccc}
\toprule
% \textbf{Features} &
% \specialcellbold{Short description} &
\specialcellbold{Combination} &
\specialcellbold{P} &
\specialcellbold{R} &
\specialcellbold{F$_1$} \\
\midrule
2 3 & .224 & .645 & .323\\
2 4 & .245 & .627 & .342\\
2 5 & .227 & .581 & .317 \\
2 6 & .217 & .651 & .317\\
2 7 & .214 & .625 & .311\\
3 4 & .252 & .577 & .339\\
2 3 4 & .265 & .607 & .357\\
\bottomrule
\end{tabular}
\end{table}
\subsection{Metrics}
% which metrics and why we used them
% cite other evaluation metrics

\subsection{Model Performances}
% explanation for good and bad performances

\subsection{Model Combinations}
% correlations and combination performances


\section{Conclusion}



\bibliographystyle{acm}
\bibliography{sigproc} 

\end{document}
