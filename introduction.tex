In the last two decades, the possibilities of discussing news articles changed radically.
One of the most significant changes is that nowadays readers are able to share their opinion about a published article directly, e.g. in a comment section.
Despite the positive aspects of this freedom it is also often misused.
In consequence comment sections of online newspapers often get moderated which binds resources and is cost intensive.
In order to schedule publication times and plan moderator team sizes, it could be helpful to know which of the published articles will get a large amount of comments before its release.

Related work is trying to predict the comment volume of newspaper articles by using a random forest approach \cite{tsagkias2009predicting}.
Also using a random forest, Ambroselli et al. defined the classification task of identifying the weekly top $10\%$ of articles with the highest comment volume \cite{ambroselli2018prediction}.

In this paper, we take the same classification task and study if a deep learning approach can outperform their model on a similar dataset.
Thereby, we consider several different models and features.
Our contributions are (a) an analysis of the influence of different article features on the prediction performance, (b) the evaluation of different model architectures, and (c) a comparison of our results against other baselines.
