We used the metrics \textit{precision} and \textit{recall} to evaluate our models. A high precision is important because every article that gets classified as an top article binds resources in a real world scenario. A high recall is important as well since we want to miss as few positives as possible. Precision and recall are equally relevant as also mentioned by Ambroselli et al. (cite) therefore we used their harmonic mean, the $F_1$-score.
As \textit{precision}, \textit{recall} and the $F_1$-score are common metrics for binary classification problems, this allows us to compare our results against existing approaches.

\subsection{Model Performances}
All our models reach a significantly higher \textit{recall} than \textit{precision}.

\textit{Model 3} reached the highest precision with a value of $0.228$. \textit{Model 7} reached the highest recall with $0.917$. The best overall performing model was \textit{Model 4} with an $F_1$-score of $0.320$. \textit{Model 1} to \textit{3} also performed relatively good with $F_1$-scores well over $0.3$.

\textit{Model 1} and \textit{Model 2} get the same input. Despite the fact that the second one has approximately a third of the weights in its hidden layer both models have a similar performance for all three metrics.
The convolutional network can learn less features but has a much higher generalization potential through its shared weights.

\textit{Model 3} takes just the first $50$ words of the article text as input because we discovered that using more or even all words has no positive effect on the model's predictions. In comparison to \textit{Model 1} and \textit{2} it reaches a worse \textit{recall} but a slightly higher \textit{precision} which leads to a similar $F_1$-score.
The performance is explainable by the similar context dimensionality the headline and the first words of an article can set.

\textit{Model 4} achieved the best results. This seems surprising due to the low complexity of this feature but the relatively good results can be justified with the usage frequency of the categories within the GACC. Articles assigned to the most used category have a chance of $22.8\%$ to become a top article. Classifying just articles within this category as a top article already leads to a $F_1$-score of $0.292$. Using just the four categories that have the most top articles in percentage terms leads to an $F_1$-score of $0.320$ which is the same result as achieved by this model.

The worst performance was achieved by \textit{Model 5}. Its precision of $0.100$ is just as good as a random prediction. A reason for the publication time performing so poorly is that $19\%$ of all articles got released on full hours and might be due to the fact that \textit{The Guardian} has an international readership spread over multiple time zones. Therefore this feature has close to no effect on the number of comments an article receives. Another reason is, that the day of the week has also no significant effect on the weekly article performance.

\textit{Model 6} had also a small but slightly better performance. Just the length of a text without context has nearly no meaning and is therefore not feasible for an accurate prediction.

\textit{Model 7} takes just a single numeric value as input. For a good performance of this model the interval of competitive scores should be divided into clear intervals that can be assigned to one class each. This is unfortunately not the case and therefore a clustering of this feature is not possible.

% Table 3: Precision (P), recall (R), and F1-score of the baseline, all article and metadata features, annotations of comments shown on the first page, and all combined.
\begin{table}[h]
	\centering
	\caption{\textmd{Precision (P), recall (R), and $F_1$-score of all basic models.}}
	\label{tbl:results_basic}
	\vspace{-0.2cm}
	\begin{tabular}{cllccc}
		\toprule
		% \textbf{Features} &
		\specialcellbold{ID} &
		\specialcellbold{Input} &
		\specialcellbold{Type} &
		\specialcellbold{P} &
		\specialcellbold{R} &
		\specialcellbold{F$_1$} \\
		\midrule
		1 & headline & FC & .216 & .606  & .309  \\
		2 & headline & CNN & .221 & .603  & .314  \\
		3 & article & LSTM  & .228 & .493  & .302  \\
		4 & category & FC & .227 & .603  & .320  \\
		5 & time & FC  & .100 & .457  & .155  \\
		6 & text metrics & FC  & .133 & .738  & .222  \\
		7 & competitive score & FC & .112 & .917  & .197  \\
		\bottomrule
	\end{tabular}
\end{table}

\subsection{Combined Models}
As seen in the correlation matrix (Figure \ref{fig:correlation_matrix}) \textit{Model 1} and \textit{Model 2} correlate the most with a correlation value of 0.589. The same feature input explains this behaviour. The correlation between \textit{Model 3} and both \textit{Model 1} and \textit{Model 2} is also relatively high with the values $0.289$ and $0.255$ respectively. A reason for this could be that a headline and the first 50 words of an article transport a very similar context.

\textit{Model 2} is the best model using text therefore we decided to combine it with each other model. For most other models the correlation with \textit{Model 2} is low which also justifies this decision.
\textit{Model 5} which is correlating least with all other models has a very low $F_1$ score. Therefore we excluded it from our consideration to combine it with every other model.

The models, that we created combining \textit{Model 2} with each \textit{Model 5}, \textit{6} and \textit{7} performed almost the same as \textit{Model 2} itself. Therefore, we figured that publishing time, text metrics and competitive score are unqualified features for the headline text.

Combining each \textit{Model 3} and \textit{Model 4} with \textit{Model 2} showed small improvements in performance. In consequence we combined \textit{model 2}, \textit{Model 3} and \textit{Model 4} and got our best results. The better $F_1$-score of $0.357$ was due to the improvements of the precision value and a consistent recall.

Our $F_1$-score was $37\%$ higher than the presented value from Tsagikas et al. \cite{tsagkias2009predicting} and $24\%$ smaller than the presented value from Ambroselli et al. \cite{ambroselli2018prediction}.

BEGRÜNDUNG \& RECHTFERTIGUNG

\begin{table}[]
\centering
\label{tbl:results_combined}
\caption{\textmd{Precision (P), recall (R), and F$_1$-score of all combined models}}
\vspace{-0.2cm}\begin{tabular}{cccc}
\toprule
% \textbf{Features} &
% \specialcellbold{Short description} &
\specialcellbold{Combination} &
\specialcellbold{P} &
\specialcellbold{R} &
\specialcellbold{F$_1$} \\
\midrule
2 3 & .224 & .645 & .323\\
2 4 & .245 & .627 & .342\\
2 5 & .227 & .581 & .317 \\
2 6 & .217 & .651 & .317\\
2 7 & .214 & .625 & .311\\
3 4 & .252 & .577 & .339\\
2 3 4 & .265 & .607 & .357\\
\bottomrule
\end{tabular}
\end{table}