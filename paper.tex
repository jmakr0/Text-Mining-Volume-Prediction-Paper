\documentclass[10pt,sigconf]{acmart}

\usepackage[utf8]{inputenc}
\usepackage{booktabs} % For formal tables
\usepackage{array}
\newcommand*\rotbf[1]{\rotatebox{90}{\textbf{#1}}}
\newcommand{\specialcell}[2][c]{\begin{tabular}[#1]{@{}l@{}}#2\end{tabular}}
\newcommand{\specialcellbold}[2][c]{%
  \bfseries
  \begin{tabular}[#1]{@{}l@{}}#2\end{tabular}%
}

% Copyright
%\setcopyright{none}
%\setcopyright{acmcopyright}
%\setcopyright{acmlicensed}
\setcopyright{rightsretained}
%\setcopyright{usgov}
%\setcopyright{usgovmixed}
%\setcopyright{cagov}
%\setcopyright{cagovmixed}




\begin{document}
\title{Comment Volume Prediction on News Articles}
\subtitle{Extended Abstract} % do we need that?

\author{Johannes M. Kroschewski}
\affiliation{%
	\institution{Hasso-Plattner-Institut,\\ University of Potsdam}
	\streetaddress{Prof.-Dr.-Helmert-Str. 2–3}
	\city{Potsdam} 
	\state{Germany} 
	\postcode{14482}
}
\email{johannes.kroschewski@student.hpi.de}

\author{Friedrich C. Schöne}
\affiliation{%
	\institution{Hasso-Plattner-Institut,\\ University of Potsdam}
	\streetaddress{Prof.-Dr.-Helmert-Str. 2–3}
	\city{Potsdam} 
	\state{Germany} 
	\postcode{14482}
}
\email{friedrich.schoene@student.hpi.de}

\author{Nils H. Straßenburg}
\affiliation{%
  	\institution{Hasso-Plattner-Institut,\\ University of Potsdam}
	\streetaddress{Prof.-Dr.-Helmert-Str. 2–3}
	\city{Potsdam} 
	\state{Germany} 
	\postcode{14482}
}
\email{nils.strassenburg@student.hpi.de}

% The default list of authors is too long for headers}
\renewcommand{\shortauthors}{M. Kroschewski et al.}


\begin{abstract}
	abstract text \footnote{abstract footnote}
\end{abstract}

%
% The code below should be generated by the tool at
% http://dl.acm.org/ccs.cfm
% Please copy and paste the code instead of the example below. 
%
\begin{CCSXML}
	% TODO
<ccs2012>
 <concept>
  <concept_id>10010520.10010553.10010562</concept_id>
  <concept_desc>Computer systems organization~Embedded systems</concept_desc>
  <concept_significance>500</concept_significance>
 </concept>
 <concept>
  <concept_id>10010520.10010575.10010755</concept_id>
  <concept_desc>Computer systems organization~Redundancy</concept_desc>
  <concept_significance>300</concept_significance>
 </concept>
 <concept>
  <concept_id>10010520.10010553.10010554</concept_id>
  <concept_desc>Computer systems organization~Robotics</concept_desc>
  <concept_significance>100</concept_significance>
 </concept>
 <concept>
  <concept_id>10003033.10003083.10003095</concept_id>
  <concept_desc>Networks~Network reliability</concept_desc>
  <concept_significance>100</concept_significance>
 </concept>
</ccs2012>  
\end{CCSXML}

\ccsdesc[500]{Computer systems organization~Embedded systems}
\ccsdesc[300]{Computer systems organization~Redundancy}
\ccsdesc{Computer systems organization~Robotics}
\ccsdesc[100]{Networks~Network reliability}

% We no longer use \terms command
%\terms{Theory}

\keywords{text mining}

\maketitle

\section{Introduction}
\subsection{Motivation and Goal}

\subsection{Related Work}


\section{Dataset}
\subsection{Description}

\subsection{Data Cleaning}

\subsection{Data Enrichment}
% e.g. competitive score


\section{Methodology}
% subsections ?


\section{Evaluation}
\begin{table}[]
\centering
\label{tbl:heatwheel_res}
\begin{tabular}{lccccc}
\toprule
% \textbf{Features} &
\specialcellbold{Input} &
\specialcellbold{Architecture} &
\specialcellbold{ID} &
\specialcellbold{P} &
\specialcellbold{R} &
\specialcellbold{F1} \\
\midrule
headline & FC & 1 & .216 & .606  & .309  \\
headline & CNN & 2 & .221 & .603  & .314  \\
article & LSTM  & 3 & .228 & .493  & .302  \\
category & FC & 4 & .227 & .603  & .320  \\
time & FC  & 5 & .100 & .457  & .155  \\
text metrics & FC  & 6 & .133 & .738  & .222  \\
competitive score & FC & 7 & .112 & .917  & .197  \\
\bottomrule
\end{tabular}
\end{table}
\begin{table}[]
\centering
\label{tbl:heatwheel_res}
\begin{tabular}{cccc}
\toprule
% \textbf{Features} &
% \specialcellbold{Short description} &
\specialcellbold{Model ID} &
\specialcellbold{P} &
\specialcellbold{R} &
\specialcellbold{F1} \\
\midrule
23 & .224 & .645 & .323\\
24 & .245 & .627 & .342\\
25 & .227 & .581 & .317 \\
26 & .217 & .651 & .317\\
27 & .214 & .625 & .311\\
34 & .252 & .577 & .339\\
234 & .265 & .607 & .357\\
\bottomrule
\end{tabular}
\end{table}

\subsection{Metrics}
% which metrics and why we used them
% cite other evaluation metrics

\subsection{Model Performances}
% explanation for good and bad performances

\subsection{Model Combinations}
% correlations and combination performances


\section{Conclusion}



\bibliographystyle{acm}
\bibliography{sigproc} 

\end{document}
